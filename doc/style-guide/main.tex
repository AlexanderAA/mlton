\documentclass[12pt]{article}
\usepackage{alltt,epsfig,html,latexsym,longtable,makeidx,moreverb}

\setlength\topmargin{-0.5in}
\setlength\textheight{8.5in}
\setlength\textwidth{7.0in}
\setlength\oddsidemargin{-0.3in}
\setlength\evensidemargin{-0.3in}
\hyphenation{}
\title{{\mlton} SML Style Guide}
\author{Stephen Weeks}
\date{\today}
\newcommand{\absolutelink}[1]{\link{\makeurl{#1}}}
\newcommand{\addr}{mlton.org}
\newcommand{\alternative}[1]{    &  |  & #1\\}
\newcommand{\alt}{\ |\ }
\newcommand{\doc}{\mbox{\tt doc/mlton}}
\newcommand{\ffi}{{\tt \_ffi}}
\newcommand{\filelink}[1]{\htmladdnormallink{{\tt #1}}{file:#1}}
\newcommand{\kit}{ML Kit}
\newcommand{\link}[1]{\htmladdnormallink{{\tt #1}}{#1}}
\newcommand{\mailto}[2]{{\tt #1 at #2}}
\newcommand{\makeurl}[1]{http://www.\addr/#1}
\newcommand{\mltonmail}{\mailto{MLton}{\addr}}
\newcommand{\mlton}{MLton}
\newcommand{\mosml}{Moscow ML}
\newcommand{\nline}{{\tt \#line}}
\newcommand{\poly}{Poly/ML}
\newcommand{\prefix}{usr}
\newcommand{\production}[2]{#1  & ::=  & #2\\}
\newcommand{\secref}[1]{Section~\ref{section:#1}}
\newcommand{\smlnj}{SML/NJ}
\newcommand{\subsec}[2]{\subsection{#1}\label{section:#2}}
\newcommand{\subsubsec}[2]{\subsubsection{#1}\label{section:#2}}
\newcommand{\version}{MLTON_VERSION}
\renewcommand{\sec}[2]{\section{#1}\label{section:#2}}

\makeindex

\begin{document}

\maketitle
This document describes how to install and use {\mlton}, the
whole-program optimizing compiler for the
\htmladdnormallink{Standard ML}
		  {http://cm.bell-labs.com/cm/cs/what/smlnj/sml.html}
programming language.
{\mlton} generates standalone executables on X86 machines with
Linux or Cygwin/Windows. 
{\mlton} is distributed under the GNU General Public License (for details, see
the {\tt MLton-LICENSE} file).

The {\mlton} homepage is \absolutelink{}.
To send {\mlton}'s authors a comment, question, or bug report, use
\mltonmail.
To send mail to the {\mlton} user community, use
\mailto{MLton-user}{\addr}.
This document describes {\mlton} version {\version}.



% conventions chosen so that inertia is towards modularity and reuse
% not to type fewer characters

\sec{High-level structure}{high-level-structure}

Code is structured in {\mlton} so that signatures are closed.  Thus, in
{\mlton}, one would never write the following.
\begin{verbatim}
signature SIG =
   sig
      val f: Foo.t -> int
   end
\end{verbatim}
Instead, one would write the following.
\begin{verbatim}
signature SIG =
   sig
      structure Foo: FOO

      val f: Foo.t -> int
   end
\end{verbatim}
The benefit of this approach is that one can first understand the
specifications (i.e. signatures) of all of the modules in {\mlton} before having
to look at any implementations (i.e. structures or functors).  That is, the
signatures are self-contained.

We deviate from this only in allowing references to top level types (like {\tt
int}), basis library modules, and {\mlton} library modules.  So, the following
signature is fine, because structure {\tt Regexp} is part of the {\mlton}
library.
\begin{verbatim}
signature SIG =
   sig
      val f: Regexp.t -> int
   end
\end{verbatim}

We also use signatures to express (some of) the dependencies between modules.
For every module {\tt Foo}, we write two signatures in a file named {\tt
foo.sig}.  The signature {\tt FOO} specifies what is implemented by {\tt Foo}.
The signature {\tt FOO\_STRUCTS} specifies the modules that are needed in order
to specify {\tt Foo}, but that are not implemented by {\tt Foo}.  As an example,
consider {\mlton}'s closure conversion pass (in {\tt mlton/closure-convert}),
which converts from {\tt Sxml}, {\mlton}'s higher-order simply-typed
intermediate language, to {\tt Cps}, {\mlton}'s first-order simply-typed
intermediate language.  The file {\tt closure-convert.sig} contains the
following.
\begin{verbatim}
signature CLOSURE_CONVERT_STRUCTS = 
   sig
      structure Sxml: SXML
      structure Cps: CPS
      sharing Sxml.Atoms = Cps.Atoms
   end

signature CLOSURE_CONVERT = 
   sig
      include CLOSURE_CONVERT_STRUCTS

      val closureConvert: Sxml.Program.t -> Cps.Program.t
   end
\end{verbatim}
These signatures say that the {\tt ClosureConvert} module implements a function
{\tt closureConvert} that transforms an {\tt Sxml} program into a {\tt Cps}
program.  They also say that {\tt ClosureConvert} does not implement {\tt Sxml}
or {\tt Cps}.  Rather, it expects some other modules to implement these and for
them to be provided to {\tt ClosureConvert}.  The sharing constraint expresses
that the ILs must share some basic atoms, like constants, variables, and
primitives.

Given the two signatures that specify a module, the module definition always has
the same structure.  A module {\tt Foo} is implemented in a file named {\tt
foo.fun}, which defines a functor named {\tt Foo} that takes as an argument a
structure matching {\tt FOO\_STRUCTS} and returns as a result a structure
matching {\tt FOO}.  For example, {\tt closure-convert.fun} contains the
following.
\begin{verbatim}
functor ClosureConvert (S: CLOSURE_CONVERT_STRUCTS): CLOSURE_CONVERT = 
struct

open S

fun closureConvert ...

end
\end{verbatim}
Although the signatures for {\tt ClosureConvert} express the dependence
on the {\tt Sxml} and {\tt Cps} ILs, they do not express the
dependence on other modules that are only used internally to closure
conversion.  For example, closure conversion uses an auxiliary module {\tt
AbstractValue} as part of its higher-order control-flow analysis.  Because {\tt
AbstractValue} is only used internally to closure conversion, it does not appear
in the signatures that specify closure conversion.  So, helper functors (like
{\tt AbstractValue}) are analogous to helper functions in that they are not
visible to clients.

We do not put helper functors lexically in scope because SML only allows top
level functor definitions and, more importantly, because files would become
unmanageably large.  Instead, helper functors get their own {\tt .sig} and {\tt
.fun} file, which follow exactly the convention above.

\section{General conventions}

\begin{itemize}
\item A line of code never exceeds 80 columns.
\item Use alphabetical order wherever possible.
\begin{itemize}
\item record field names
\item datatype constructors
\item value specs in signatures
\item file lists in CM files
\item export lists in CM files
\end{itemize}
\end{itemize}

%------------------------------------------------------
%                Signature conventions                 
%------------------------------------------------------

\sec{Signatures}{signature-conventions}

We now enumerate the conventions we follow in writing signatures.

\begin{enumerate}

\item
Signature identifiers are in all capitals, using ``\_'' to
separate words.

\item
A signature typically contains a single type specification that defines a type
constructor {\tt t}, which is the type of interest in the specification.  For
oexample, here are signature fragments for integers, lists, and maps.
\begin{verbatim}
signature INTEGER =
   sig
      type t

      val + : t * t -> t
      ...
   end

signature LIST =
   sig
      type 'a t

      val map: 'a t * ('a -> 'b) -> 'b t
      ...
   end

signature MAP
   sig
      type ('a, 'b) t

      val extend: ('a, 'b) t * 'a * 'b -> ('a, 'b) t
      ...
   end
\end{verbatim}
Although at first it might appear confusing to name every type {\tt t}, in fact
there is never ambiguity, because at any point in the program there is at most
one unqualified {\tt t} in scope, and all other types will be named with long
identifiers (like {\tt Int.t} or {\tt Int.t List.t}).  For example, the code for
a function {\tt foo} within the {\tt Map} module might look like the following.
\begin{verbatim}
fun foo (l: 'a List.t, n: Int.t): ('a, Int.t) t = ...
\end{verbatim}

In practice, for pervasive types like {\tt int}, {\tt 'a list}, we often use the
standard pervasive name instead of the {\tt t} name.

\item Signatures should not contain free types or structures, other than
pervasives, basis library modules, or {\mlton} library modules.  This was
explained in \secref{high-level-structure}.

\item
If additional abstract types (other than pervasive types) are needed to specify
operations, they are included as substructures of the signature, and have a
signature in their own right. For example, the following signature is good.

\begin{verbatim}
signature FOO =
   sig
      structure Var: VAR

      type t
      val fromVar: Var.t -> t
      val toVar: t -> Var.t
   end
\end{verbatim}

\item
Signatures do not use substructures or multiple structures to group different
operations on the same type.  This makes you waste energy remembering where the
operations are.  For exmample, the following signature is bad.

\begin{verbatim}
signature REAL =
   sig
     type t

     val + : t * t -> t

     structure Trig:
        sig
           val sin: t -> t
           val cos: t -> t
        end
   end
\end{verbatim}

\item
Signatures usually should not contain datatypes.  This exposes the
implementation of what should be an abstract type.  For example, the following
signature is bad.
\begin{verbatim}
signature COMPLEX =
   sig
      datatype t = T of real * real
   end
\end{verbatim}
A common exception to this rule is abstract syntax trees.

\item
Use structure sharing to express type sharing.  For example, in {\tt
closure-convert.sig}, a single structure sharing equation expresses a number of
type sharing equations.

\end{enumerate}

%------------------------------------------------------
%                 Value specifications                 
%------------------------------------------------------

\subsec{Value specifications}{val-specs}

Here are the conventions that we use for individual value specifications in
signatures.  Of course, many of these conventions directly impact the way in
which we write the core language expressions that implement the specifications.

\begin{enumerate}

\item
In a datatype specification, if there is a single constructor, then that
constructor is called {\tt T}.
\begin{verbatim}
datatype t = T of int
\end{verbatim}

\item
In a datatype specification, if a constructor carries multiple values of the
same type, use a record to name them to avoid confusion.
\begin{verbatim}
datatype t = T of {length: int, start: int}
\end{verbatim}

\item
Identifiers begin with and use small letters, using capital letters to separate
words.
\begin{verbatim}
val helloWorld: unit -> unit
\end{verbatim}

\item
There is no space before the colon, and a single space after it.  In the case of
operators (like {\tt +}), there is a space before the colon to avoid lexing the
colon as part of the operator.

\item
Pass multiple arguments as tuple, not curried.
\begin{verbatim}
val eval: Exp.t * Env.t -> Val.t
\end{verbatim}

\item
Currying is only used when there staging of a computation, i.e., if
precomputation is done on one of the arguments.
\begin{verbatim}
val match: Regexp.t -> string -> bool
\end{verbatim}

\item
Functions which take a single element of the abstract type of a signature take
the element as the first argument, and auxiliary arguments after.
\begin{verbatim}
val push: t * int -> unit
val map: 'a t * ('a -> 'b) -> 'b t
\end{verbatim}

\item
$n$-ary operations take the $n$ elements first, and auxilary arguments after.
\begin{verbatim}
val merge: 'a t * 'a t * ('a * 'a -> 'b) -> 'b t
\end{verbatim}

\item
If two arguments to a function are of the same type, and the operation is not
commutative, pass them using a record.  This names the arguments and ensures
they are not confused.  Exceptions are the standard numerical and algebraic
operators.
\begin{verbatim}
val fromTo: {start: int, step: int, stop: int} -> int list
val substring: t * {length: int, start: int} -> t
val - : t * t -> t
\end{verbatim}

\item
Field names in record types are written in alphabetical order.

\item
Return multiple results as a tuple, or as a record if there is the potential for
confusion.
\begin{verbatim}
val parse: string -> t * string
val quotRem: t * t -> t * t
val partition: 'a t * ('a -> bool) -> {no: 'a t, yes: 'a t}
\end{verbatim}

\item
If a function returns multiple results, at least two of which are of the same
type, and the name of the function does not clearly indicate which result is
which, use a record to name the results.
\begin{verbatim}
val vars: t -> {frees : Vars.t, bound : Vars.t}
val partition: 'a t * ('a -> bool) -> {yes : 'a t, no : 'a t}
\end{verbatim}

\item
Use the same names and argument orders for similar functions in different
signatures.  This is especially common in the {\mlton} library.
\begin{verbatim}
val < : t * t -> bool
val equals: t * t -> bool
val forall: 'a t * ('a -> bool) -> bool
\end{verbatim}

\item
Use {\tt is}, {\tt are}, {\tt can}, etc. to name predicates.  One exception is
{\tt equals}.
\begin{verbatim}
val isEven: int -> bool
val canRead: t -> bool
\end{verbatim}

\end{enumerate}

%------------------------------------------------------
%                  Signature example                   
%------------------------------------------------------

\subsection{Example}

Here is the complete specification of a simple interpreter.  This demonstrates
the {\tt t}-convention, the closed-signature convention, and the use of sharing
constraints.

\begin{verbatim}
signature VAR =
   sig
      type t
   end

signature EXP =
   sig
      structure Var: VAR

      datatype t =
         Var of Var.t
       | Lam of Var.t * t
       | App of t * t
   end

signature VAL =
   sig
      structure Var: VAR

      type t

      val var: Var.t -> t
      val lam: Var.t * t -> t
      val app: t * t -> t
   end

signature INTERP =
   sig
      structure Exp: EXP
      structure Val: VAL
      sharing Exp.Var = Val.Var

      val eval: Exp.t -> Val.t
   end

signature ENV =
   sig
      structure Var: VAR

      type 'a t

      val lookup: 'a t * Var.t -> 'a
      val extend: 'a t * Var.t * 'a -> 'a t
   end
\end{verbatim}

%------------------------------------------------------
%               Functors and structures                
%------------------------------------------------------

\section{Functors and structures}
We now enumerate the conventions we follow in writing functors and structures.
There is some repetition with \secref{high-level-structure}.

\begin{enumerate}

\item
Functor identifiers begin with capital letters, use mixed case, and use capital
letters to separate words.

\item
Functor definitions look like the following.
\begin{verbatim}
functor Foo (S: FOO_STRUCTS): FOO =
struct

open S

...

end
\end{verbatim}

\item
The name of the functor is the same as the name of the signature describing the
structure it produces.

\item
The functor result is constrained by a signature.

\item
A functor takes as arguments any structures that occur in the signature of the
result that it does not implement.

\item
Structure identifiers begin with capital letters, and use capital letters to
separate words.

\item
The name of the structure is the same as the name of the functor that produces
it.

\item
A structure definition looks like one of the following.
\begin{verbatim}
structure Foo = Foo (S)

structure Foo =
   struct
      ...
   end
\end{verbatim}

\item
Avoid the use of {\tt open} except within tightly constrained scopes.  The use
of {\tt open} makes it hard to look at code later and understand where things
come from.

\end{enumerate}

%------------------------------------------------------
%                   Core expressions                   
%------------------------------------------------------

\section{Core expressions}

We now enumerate the conventions we follow in writing core expressions.  We do
not repeat the conventions of \secref{val-spec}, although many of them apply
here.
\begin{enumerate}

\item
Tuples are written with spaces after commas, like {\tt (a, b, c)}.

\item
Records are written with spaces on both sides of equals and with spaces after
commas, like {\tt \{bar = 1, foo = 2\}}.

\item
Record field names are written in alphabetical order, both in expressions and
types.

\item
Function application is written with a space between the function and the
argument.  If there is one untupled argument, it looks like {\tt f x}.  If there
is a tupleg argument, it looks like {\tt f (x, y, z)}.

\item
When you want to mix declarations with side-effecting statements, use a
declaration like {\tt val \_ = sideEffectingProcedure()}.

\item
In sequence expressions {\tt (e1; e2)} that span multiple lines, place the
semicolon at the beginning of lines.
\begin{verbatim}
(e1
 ; e2
 ; e3)
\end{verbatim}

\item
Never write nonexhaustive matches.  Always handle the default case and raise an
error message.  Your error message will be better than the compiler's.  Also, if
you have lots of uncaught cases, then you are probably not using the type system
in a strong enough way - your types are not expressing as much as they could.

\item
Never use the syntax for declaring functions that repeats the function name.
Use {\tt case} or {\tt fn} instead.  That is, do not write the following.
\begin{verbatim}
fun f 0 = 1
  | f n = n + 1
\end{verbatim}
Instead, write the following.
\begin{verbatim}
val f =
   fn 0 => 1
    | n => n + 1
\end{verbatim}
Or, write the following.
\begin{verbatim}
fun f n =
   case n of
      0 => 1
    | _ => n + 1
\end{verbatim}

\end{enumerate}

\bibliographystyle{alpha}
\bibliography{bib}
\end{document}
