\sec{Compiling {\mlton}}{compiling}

If you want to compile {\mlton}, you need either the source {\tt rpm}
or {\tt tgz}.  You can compile with either {\mlton} or {\smlnj}, but
we strongly recommend using {\mlton}, since it generates a much faster
executable.

\subsection{Compiling with {\mlton}}

To compile with {\mlton}, you need the binary versions of {\tt mlton},
{\tt mllex}, and {\tt mlyacc} that come with the {\mlton} binary
package.  An older, the same, or a newer version of {\mlton} should
work.  To build {\mlton}, run {\tt make} from within the root
directory of the sources.  This will make the {\tt bootstrap} target,
which will build {\mlton} first with the already installed binary
version of {\mlton} and will then rebuild {\mlton} with itself.

First, the Makefile calls {\tt mllex} and {\tt mlyacc} to build the
lexer and parser, and then calls {\tt mlton} to compile itself.  When
making {\mlton} using another version the Makefile automatically uses
{\tt mlton-stubs.cm}, which will put in enough stubs to emulate the
{\tt MLton} structure.  Once {\mlton} is built, the Makefile will
rebuild {\mlton} with itself, this time using {\tt mlton.cm} and the
real {\tt MLton} structure from the basis library.  This second round
of compilation is essential in order to achieve a fast and robust
{\mlton}.

Compiling {\mlton} requires at least 256M of actual RAM.  Thus, if
your machine has less than this, it is likely that self-compilation
will take a very long time due to paging.  Even if you have enough
memory, there simply may not be enough available, due to memory
consumed by other processes.  In this case, you may see an {\tt Out of
memory} message, or self-compilation may become extremely slow.  The
only fix is to make sure that enough memory is available.

\subsection{Compiling with {\smlnj}}

To compile with {\smlnj}, run {\tt make nj-mlton} from within the root
directory of the sources.  First, the Makefile calls {\tt mllex} and
{\tt mlyacc} to build the lexer and parser.  Then, it calls {\smlnj}
with the appropriate {\tt sources.cm} file.  Building with {\smlnj}
takes some time (5-10 minutes on a 1.6GHz machine).  Unless you are
doing compiler development and need rapid recompilation, we recommend
compiling with {\mlton}.
