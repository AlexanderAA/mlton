\sec{Compiling {\mlton}}{compiling}

If you want to compile {\mlton}, you need either the source {\tt rpm} or {\tt
tgz}.  You can compile with either {\mlton} or {\smlnj}.

\subsection{Compiling with {\mlton}}

To compile with {\mlton}, you need binary versions of {\tt mlton},
{\tt mllex}, and {\tt mlyacc} that come with a {\mlton} binary {\tt
rpm} or {\tt tgz} (an older or newer version of {\mlton} should work).
To compile, run {\tt make} from within the root directory directory of the
sources.
\begin{verbatim}
% make
mlyacc front-end/ml.grm
1 shift/reduce conflict
chmod -w front-end/ml.grm.*
rm -f front-end/ml.lex.sml
mllex front-end/ml.lex

Number of states = 329
Number of distinct rows = 216
Approx. memory size of trans. table = 222912 bytes
chmod -w front-end/ml.lex.sml
Compiling mlton (takes a while)
time mlton @MLton  gc-summary -- -host self -verbose 1 -output mlton-compile mlton.cm
MLton starting
   Compile SML starting
      pre codegen starting
...
\end{verbatim}
This calls {\tt mlyacc} and {\tt mllex} to build the parser and lexer, and then
calls {\tt mlton} to compile itself.  If you make {\mlton} using another version
of {\mlton}, then the Makefile will automatically use {\tt mlton-stubs.cm},
which will put in enough stubs to emulate the {\tt MLton} structure.  Once you
have successfully built {\mlton}, you should recompile it with itself, which
will use {\tt mlton.cm} to bypass the stubs and use the real {\tt MLton}
structure from the basis library.

Compiling {\mlton} requires at least 256M of actual RAM.  Thus, if your machine
has less than this, it is likely that self-compilation will take a very long
time due to paging.  Even if you have enough memory, there simply may not be
enough available, due to memory consumed by other processes.  In this case, you
may see an ``Out of memory'' message, or self-compilation may become extremely
slow.  The only fix is to make sure that enough memory is available.

\subsection{Compiling with {\smlnj}}

To compile with {\smlnj} version 110.39 or later, do {\tt make nj-mlton} from
within the root directory of the sources.
\begin{verbatim}
% make nj-mlton
make dirs
make[1]: Entering directory `/home/sweeks/mlton/src'
mkdir -p /home/sweeks/mlton/src/build/bin /home/sweeks/mlton/src/build/lib/self/include
make[1]: Leaving directory `/home/sweeks/mlton/src'
cd /home/sweeks/mlton/src/mlton && make nj-mlton
make[1]: Entering directory `/home/sweeks/mlton/src/mlton'
(									\
	echo 'SMLofNJ.Internals.GC.messages false;';			\
	echo '#set CM.Control.verbose false;';				\
	echo '#set CM.Control.warn_obsolete false;';			\
	echo 'Control.polyEqWarn := false;';				\
	echo 'CM.make "sources.cm";';					\
	echo 'Main.exportNJ("/home/sweeks/mlton/src/basis-library", "/home/sweeks/mlton/src/build/lib/mlton");'	\
) | time sml
Standard ML of New Jersey v110.39 [FLINT v1.5], February 15, 2002
...
Creating constants file.
/home/sweeks/mlton/src/build/bin/mlton -build-constants >tmp.c
/home/sweeks/mlton/src/build/bin/mlton -o tmp tmp.c
./tmp >/home/sweeks/mlton/src/build/lib/self/constants
rm -f tmp tmp.c
make[1]: Leaving directory `/home/sweeks/mlton/src'
Build of MLton succeeded.
\end{verbatim}
