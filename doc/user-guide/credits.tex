\sec{Credits}{credits}

{\mlton} was designed and implemented by Henry Cejtin, Matthew Fluet, Suresh
Jagannathan, and Stephen Weeks.

\begin{itemize}

\item
Henry Cejtin (\mailto{henry}{sourcelight.com}) wrote the {\tt IntInf}
implementation, the original profiler, the original man pages, the {\tt
.spec} files for the RPMs, and lots of little hacks to speed stuff up.

\item
Matthew Fluet (\mailto{fluet}{cs.cornell.edu}) implemented the
{\intel} native code generator, ported {\tt mlprof} to work with the
native code generator, did a lot of work on the SSA optimizer, both
adding new optimizations and improving or porting existing
optimizations, and updated the basis library implementation.

\item
Suresh Jagannathan (\mailto{suresh}{cs.purdue.edu}) implemented
some early inlining and uncurrying optimizations.

\item
Stephen Weeks (\mailto{sweeks}{sweeks.com}) implemented most of the original
version of {\mlton}, and continues to keep his fingers in most every part.

\end{itemize}
Many people have helped us over the years.  Here is an alphabetical list.

\begin{itemize}

\item
\'{A}NOQ of the Sun (\mailto{anoq}{HardcoreProcessing.com}) implemented {\tt
BinIO}, modified MLton so it could cross compile to MinGW, and provided useful
discussion about cross-compilation.

\item
Alain Deutsch (\mailto{deutsch}{polyspace.com}) and \htmladdnormallink{PolySpace
Technologies}{http://www.polyspace.com/} provided many bug fixes and
runtime system improvements, as well as some code to help the Sparc
port.

\item
Simon Helsen (\mailto{shelsen}{acm.org}) has provided bug reports, suggestions,
and helpful discussions.

\item
Joe Hurd (\mailto{joe.hurd}{cl.cam.ac.uk}) provided useful discussion
and feedback on source-level profiling.

\item
Richard Kelsey (\mailto{kelsey}{research.nj.nec.com}) provided helpful
discussions.

\item
Jesper Louis Andersen (\mailto{jlouis}{mongers.org}) sent several patches
to improve the runtime on FreeBSD and ported MLton to run on NetBSD.

\item
Tom Murphy (\mailto{twm}{andrew.cmu.edu}) wrote the original version
of {\tt MLton.Syslog} as part of his {\tt mlftpd} project, provided
the description of how to type check code using {\tt \_ffi} with
{\smlnj}, and has sent many useful bug reports and suggestions.

\item
Michael Neumann (\mailto{uu9r}{rz.uni-karlsruhe.de}) helped to patch the runtime
to compile under FreeBSD.

\item
Barak Pearlmutter (\mailto{bap}{cs.unm.edu}) built the original
\htmladdnormallink{Debian packages}{http://packages.debian.org/mlton}
for {\mlton}, and helped us to take over the process.

\item
Sam Rushing (\mailto{rushing}{nightmare.com}) ported {\mlton} to FreeBSD.

\item
Jeffrey Mark Siskind (\mailto{qobi}{purdue.edu}) provided helpful discussions
and inspiration with his Stalin Scheme compiler.

\end{itemize}
We have also benefited from other software development tools and borrowed code
from other sources.

\begin{itemize}

\item
{\mlton} was developed using
\htmladdnormallink{Standard ML of New Jersey}{http://www.smlnj.org/}
and the
\htmladdnormallink{Compilation Manager (CM)}
		  {http://www.smlnj.org/doc/CM/index.html}.

\item
{\mlton}'s lexer ({\tt mlton/frontend/ml.lex}), 
parser ({\tt mlton/frontend/ml.grm}),
and precedence-parser ({\tt mlton/elaborate/precedence-parse.fun})
are modified versions of code from the {\smlnj}.

\item
{\mlton} uses the {\smlnj} library implementation of splay trees.

\item
The {\mlton} basis library implementation of {\tt Real.}\{{\tt fmt}, {\tt
fromDecimal}, {\tt fromString}, {\tt scan}, {\tt toDecimal}, {\tt
toString}\} uses David Gay's
\htmladdnormallink{gdtoa}
		  {http://www.netlib.org/fp/}
library.

\item
The {\mlton} basis library implementation uses modified versions of
portions of the the {\smlnj} basis library modules {\tt OS.IO}, {\tt
Posix.IO}, {\tt Process}, and {\tt Unix}.

\item
The {\mlton} basis library implementation uses modified versions of
portions of the
\htmladdnormallink{ML Kit Version 3}
		  {http://www.itu.dk/research/mlkit/kit3/readme.html}
basis library modules {\tt Path}, {\tt Time}, and {\tt Date}.

\item
Many of the benchmarks come from the {\smlnj} benchmark suite.

\item
Many of the regression tests come from the ML Kit Version 3 distribution, which
borrowed them from the
\htmladdnormallink{Moscow ML}
		  {http://www.dina.kvl.dk/~sestoft/mosml.html}
distribution.

\item
{\mlton} uses the
\htmladdnormallink{GNU multiprecision library}
		  {http://www.gnu.org/software/gmp/gmp.html}
for its implementation of {\tt IntInf}.

\end{itemize}
