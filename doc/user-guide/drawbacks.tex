\sec{Drawbacks of {\mlton}}{drawbacks}

\newcommand{\drawback}[1]{\item[\bf #1]\hspace{1in}\\}

\begin{description}

\drawback{poor front-end type checking}
There is quite a bit of front-end type checking that is not done.  If
you feed {\mlton} a type-incorrect program, it might produce a terse
error message and exit, it might fail to terminate, or it might
terminate and produce an executable.  Note that this does not mean
that the behavior of an executable generated by {\mlton} is random.
It simply means that the compiler may behave strangely for invalid SML
programs.  The behavior of {\mlton} for valid SML programs is well
defined.

It is strongly recommended that you type check your programs with
another SML compiler before compiling them with {\mlton}.

\drawback{large run-time memory requirement}
{\mlton}'s runtime system uses a simple two-space stop-and-copy
collector.  By default, the runtime system automatically resizes the
heap and stack.  The collector works well for programs that allocate a
lot of ephemeral data, but not so well for programs with large
long-lived data.  Because of the collector, programs often require
more memory than with other SML compilers.  There are compile-time and
run-time options available to limit the heap usage of programs.  See
\secref{manual-page} for details.

\drawback{large compile-time memory requirement}
Because {\mlton} is self-hosting, when you are compiling using
{\mlton}, you are also running a {\mlton} generated executable.
Because of the large run-time memory requirement (see above) and the demands of
whole-program compilation, the compiler requires a large amount of memory while
compiling.

\drawback{long compile times}
Whole-program compilation often takes a long time.  For example, compiling
the compiler (83K lines) on an 733 MhZ machine takes roughly ten minutes.

\drawback{no interactive top level}
Because of whole-program compilation, {\mlton} does not provide an
interactive top level.  In particular, it does not implement the {\tt use} basis
library function.

\end{description}
