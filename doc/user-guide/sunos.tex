\subsec{Running on SunOS}{sunos}

\begin{itemize}

\item
When running on Sparc's, {\mlton} uses {\tt -align 8} by default.
While this speeds up reals, it also may increase object sizes.  If
your program does not make significant use of reals, you might see a
speedup with {\tt -align 4}.

\end{itemize}

Here are the known problems using {\mlton} on SunOS.

\begin{itemize}

\item {\mlton} only supports the C code generator when running on
SPARCs. So, performance is not as good as it might be.  Compile times
are also longer.

\end{itemize}

Here are the known problems building {\mlton} on SunOS.

\begin{itemize}

\item You must install the {\tt binutils}, {\tt gcc}, and {\tt make}
packages.  You can find out how to get these at
\htmladdnormallink{Sunfreeware.com}{http://www.sunfreeware.com}.

\item Bootstrapping is so slow as to be impractical (many hours on a
500MhZ UltraSPARC).  For this reason, we strongly recommend building
with a Linux to SunOS cross compiler (\secref{cross-compiling}).

\item Making the documentation requires that you install {\tt latex}
and {\tt dvips}, which are available in the {\tt tetex} package.  It
also requires {\tt latex2html}, which we haven't yet tracked down a
package for yet.

\end{itemize}
