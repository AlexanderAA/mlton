\sec{Profiling}{profiling}

With {\mlton} and {\tt mlprof}, you can profile your program to find
out for each function how much time it spends or how many bytes it
allocates.  To profile your program, compile with either {\tt -profile
alloc} or {\tt -profile time} (you can not do both simultaneously).
Then, run the executable.  While it runs, the profiler maintains
counts (bytes or clock ticks) for each source function in the program.
When the program finishes, it automatically writes the counts to an
{\tt mlmon.out} file.  You can then run {\tt mlprof} on the executable
and the {\tt mlmon.out} file to see the percentage of the total
(allocation or time) spent in each functions.

Here is an example of time profiling, run from within the {\tt
examples/profiling} directory.
\begin{verbatim}
% mlton -profile time tak.sml
% ./tak
% mlprof tak mlmon.out
6.36 seconds of CPU time (0.0 seconds GC)
function  cur 
-------- -----
tak2     77.2%
tak1     22.8%
<gc>      0.0%
\end{verbatim}
This is a contrived example with two mutually recursive copies of the
{\tt tak} function.  The profiling shows us that roughly
three-quarters of the time is spent in the {\tt tak2} function, while
the rest is spent in {\tt tak1}.  There is a negligible amount of time
spent in gc.

You can display raw counts in addition to percentages with {\tt -raw
true}.
\begin{verbatim}
% mlprof -raw true tak mlmon.out
6.36 seconds of CPU time (0.0 seconds GC)
function  cur    raw  
-------- ----- -------
tak2     77.2% (4.91s)
tak1     22.8% (1.45s)
<gc>      0.0%  (0.0s)
\end{verbatim}

You can display the filename and line numbers for functions in addition
to their names with {\tt -show-line true}.
\begin{verbatim}
% mlprof -show-line true tak mlmon.out
6.36 seconds of CPU time (0.0 seconds GC)
   function      cur 
--------------- -----
tak2 tak.sml: 8 77.2%
tak1 tak.sml: 1 22.8%
<gc>             0.0%
\end{verbatim}

Allocation profiling is very similar to time profiling.  Here is an
example run from within the {\tt examples/profiling} directory.

\begin{verbatim}
% mlton -profile alloc list-rev.sml
% ./list-rev
% mlprof -show-line true -thresh 1 list-rev mlmon.out
63,176 bytes allocated (4,672 bytes by GC)
             function                cur 
----------------------------------- -----
append list-rev.sml: 1              87.5%
<gc>                                 6.9%
tabulate <basis>/list/list.sml: 102  3.5%
rev list-rev.sml: 6                  1.8%
\end{verbatim}

The data shows that most of the allocation is done by the append
function, defined on line 1 of {\tt list-rev.sml}.  The allocation by
the garbage collector is due to it growing the stack.  Basis library
functions are displayed with the {\tt <basis>} prefix.  The example
also shows how to filter out functions below a certain percentage with
{\tt -thresh}.

Time profiling typically has a very small performance impact.
However, the performance impact of allocation profiling is noticeable,
because it inserts additional C calls for object allocation.

\subsection{Profiling the stack}

For both allocation and time profiling, you can use {\tt
-profile-stack true} to count the time spent (or bytes allocated)
while a function is on the stack.  Here is an example.

\begin{verbatim}
% mlton -profile alloc -profile-stack true list-rev.sml
% ./list-rev
% mlprof -show-line true -thresh 1 list-rev mlmon.out
63,176 bytes allocated (4,672 bytes by GC)
             function                cur  stack  GC 
----------------------------------- ----- ----- ----
<main>                               0.2% 93.1% 6.0%
rev list-rev.sml: 6                  1.8% 87.6% 5.8%
append list-rev.sml: 1              87.5% 87.5% 3.0%
tabulate <basis>/list/list.sml: 102  3.5%  3.5% 0.0%
dot list-rev.dot >list-rev.ps
\end{verbatim}

In the above table, we see that {\tt rev}, defined on line 6 of {\tt
list-rev.sml}, is on the stack while 87.6\% of the allocation is done
by the user program and while 5.8\% of the allocation is done by the
garbage collector.  The above table also shows how special functions
like {\tt main} are handled: they are printed with surrounding
brackets.  C functions are displayed similarly.  From the above we can
also see that with {\tt -profile-stack true}, the functions are sorted
in decreasing order of bytes allocated while the function was on the
stack. With {\tt -profile-stack false}, the functions are sorted in
decreasing order of bytes allocated by that function.

For easier visualization of data, {\tt mlprof} also creates a call
graph of the program, in {\tt .dot} format, containing the table data.
You can create a postscript graph from the dot file using the
\htmladdnormallink{{\tt graphviz}}
		  {http://www.research.att.com/sw/tools/graphviz/}
software package.  Technically speaking, the graph is a call-stack
graph rather than a call graph because it describes the set of
possible call stacks.  Thedifference is in how tail calls are
displayed.  For example if {\tt f} nontail calls {\tt g} and {\tt g}
tail calls {\tt h}, then the call-stack graph has edges from {\tt f}
to {\tt g} and {\tt f} to {\tt h}, while the call-graph would have
edges from {\tt f} to {\tt g} and {\tt g} to {\tt h}.

The performance impact of {\tt -profile-stack true} can be noticeable
since there is some extra bookkeeping at every nontail call.

\subsection{Using {\tt MLton.Profile}}

To profile individual portions of your program, you can use the {\tt
MLton.Profile} structure (see \secref{profile-structure}).  This
allows you to create many units of profiling data (essentially,
mappings from functions to counts) during a run of a program, to
switch between them while the program is running, and to output
multiple {\tt mlmon.out} files.  Executing {\tt mlprof} with multiple
{\tt mlmon.out} files sums the profiling data in each file to produce
the output profiling information.

Here is an example, run from within the {\tt examples/profiling}
directory, showing how to profile the executions of the {\tt fib} and
{\tt tak} functions separately.

\begin{verbatim}
% mlton -profile time fib-tak.sml
% ./fib-tak
% mlprof fib-tak mlmon.fib.out
5.67 seconds of CPU time (0.0 seconds GC)
function   cur 
--------- -----
fib       96.8%
<unknown>  3.2%
<gc>       0.0%
% mlprof fib-tak mlmon.tak.out
0.72 seconds of CPU time (0.0 seconds GC)
function  cur  
-------- ------
tak      100.0%
<gc>       0.0%
% mlprof fib-tak mlmon.fib.out mlmon.tak.out mlmon.out
6.39 seconds of CPU time (0.0 seconds GC)
function   cur 
--------- -----
fib       85.9%
tak       11.3%
<unknown>  2.8%
<gc>       0.0%
\end{verbatim}

\subsection{Profiling details}

Conceptually, both allocation and time profiling work in the same way.
The compiler produces information that maps machine code positions to
source functions that the profiler uses while the program is running
to find out the current source function.  With {\tt -profile-stack
true}, the profiler also keeps track of which functions are on the
stack.  The profiler associates counters (either clock ticks or byte
counts) with source functions.  For allocation profiling, the compiler
inserts code whenever an object is allocated to call a C function to
increment the appropriate counter.  For time profiling, the profiler
catches the {\tt SIGPROF} signal 100 times per second and increments
the appropriate counter.  Then, when the program finishes, the
profiler writes the counts out to the {\tt mlmon.out} file.  Then,
{\tt mlprof} uses source information stored in the executable to
associate the counts in the {\tt mlmon.out} file with source
functions.

With {\tt -profile time}, use of the following in your program will
cause a run-time error, since they would interfere with the profiler
signal handler.
\begin{tabular}{l}
\tt MLton.Itimer.set (MLton.Itimer.Prof, ...)\}\\
\tt MLton.Signal.setHandler (MLton.Signal.prof, ...)
\end{tabular}
Also, because of the random sampling used to implement {\tt -profile
time}, it is best to have a long running program (at least tens of
seconds) in order to get reasonable time data.

There may be a few missed clock ticks or bytes allocated at the
very end of the program after the data is written.

For both forms of profiling, if your program calls {\tt
Posix.Process.exit}, you will bypass the code responsible for writing
out the profiling data and thus get no {\tt mlmon.out} file.

Profiling has not been tested with threads.

Because {\mlton} duplicates functor bodies, sometimes the same function will
appear multiple times in the profiling output.
