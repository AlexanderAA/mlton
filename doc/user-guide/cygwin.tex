\subsec{Running on Cygwin/Windows}{cygwin}

{\mlton} uses the \htmladdnormallink{Cygwin}{http://www.cygwin.com/}
emulation layer to provide a Posix-like environment while running on a
Windows machine.  To run {\mlton} on Windows, you must first install
Cygwin on your machine.  To do this, visit the Cygwin site from your
Windows machine and run their {\tt setup.exe} script.  Then, you can
unpack the {\mlton} binary tgz in your Cygwin environment.  This
version of {\mlton} was built against the Cygwin 1.3.17 header files.

To run {\mlton} cross-compiled executables on Windows, you must
install the Cygwin {\tt dll} on the Windows machine.

Here are the known problems using {\mlton} on Cygwin.

\begin{itemize}

\item Time profiling is disabled.

\item Due to several bugs in Cygwin's emulation of {\tt fork}, {\tt
Posix.Process.fork} is disabled.  Any use of {\tt fork} will raise
{\tt OS.SysErr}.  For idiomatic uses of {\tt fork} plus {\tt exec},
you can instead use the {\tt MLton.Process.spawn} family of functions, which
work on both Cygwin and Linux.

\item We have seen some strangeness in Cygwin's emulation of signals and
signal handlers, but have not been able to pin it down.

\end{itemize}
