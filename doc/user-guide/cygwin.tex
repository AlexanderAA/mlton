\subsec{Running on Cygwin/Windows}{cygwin}

{\mlton} uses the \htmladdnormallink{Cygwin}{http://www.cygwin.com/}
emulation layer to provide a Posix-like environment while running on a
Windows machine.  To run {\mlton} on Windows, you must first install
Cygwin on your machine.  To do this, visit the Cygwin site from your
Windows machine and run their {\tt setup.exe} script.  Then, you can
unpack the {\mlton} binary tgz in your Cygwin environment.

To run {\mlton} cross-compiled executables on Windows, you must
install the Cygwin {\tt dll} on the Windows machine.

Here are the known problems using {\mlton} on Cygwin.

\begin{itemize}

\item Time profiling is disabled.

\item
{\tt Posix.Process.fork} is disabled.  Any use of {\tt fork} will
raise {\tt OS.SysErr}.  For idiomatic uses of {\tt fork} plus {\tt
exec}, you can instead use the {\tt MLton.Process.spawn} family of
functions, which work on both Cygwin and Linux.

Fork used to be disabled due to Cygwin bugs involving the interaction
of fork and mmap.  Recently, Cygwin developers have fixed those bugs,
in Cygwin versions from 2004-Jul-15 or later (1.5.11-1 or higher).
However, Cygwin's mmap emulation does not make available as much
contiguous virtual address space as using the Windows {\tt
VirtualAlloc} function.  Hence, {\mlton} still uses {\tt VirtualAlloc}
and not mmap, which means that Cygwin can not properly emulate fork.
Consequently, {\tt Posix.Process.fork} must be disabled.

\end{itemize}
