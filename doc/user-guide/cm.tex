\sec{CM}{cm}

For porting code from SML/NJ and for developing code for MLton under SML/NJ,
{\mlton} supports a very limited subset of
\htmladdnormallink{Compilation Manager}
		  {http://www.smlnj.org/doc/CM/index.html}
(CM) files.  From {\mlton}'s point of view, a CM file {\tt foo.cm} defines a
list of SML source files.  The call {\tt mlton foo.cm} is equivalent to
compiling an SML program consisting of the concatenation of these files.  As
always with {\mlton}, the concatenation must be the whole program you wish to
compile.

In its simplest form, a CM file contains the keywords {\tt Group is} followed by 
an explicit list of sml files.  For example, if {\tt foo.cm} contains
\begin{verbatim}
Group is
bar.sig
bar.fun
main.sml
\end{verbatim}
then a call {\tt mlton foo.cm} is equivalent to concatenating the three files
together and calling {\mlton} on that SML file.  The list of files defined by a
CM file is the same as the order in which the filenames appear in the CM file.
Thus, order in a CM file matters.  In the above example, if {\tt main.sml}
refers to a structure defined in {\tt bar.fun}, then {\tt main.sml} must appear
after {\tt bar.fun} in the file list.

CM files can also refer to other CM files.  A reference to {\tt bar.cm} from
within {\tt foo.cm} means to include all of the SML files defined by {\tt
bar.cm} before any of the subsequent files in {\tt foo.cm}.  For example if {\tt
foo.cm} contains
\begin{verbatim}
Group is
bar.cm
main.sml
\end{verbatim}
and {\tt bar.cm} contains
\begin{verbatim}
Group is
bar.sig
bar.fun
\end{verbatim}
then a call to {\tt mlton foo.cm} is equivalent to compiling the concatenation
of {\tt bar.sig}, {\tt bar.fun}, and {\tt main.sml}.

CM also has a preprocessor mechanism that allows files to be conditionally
included.  This can be useful when developing code in {\smlnj} and {\mlton}.
In {\smlnj}, the preprocessor defines the symbol {\tt SMLNJ\_VERSION}.  In
{\mlton}, no symbols are defined.  So, to conditionally include {\tt foo.sml}
when compiling under {\smlnj}, one can use the following pattern.
\begin{verbatim}
# if (defined(SMLNJ_VERSION))
foo.sml
# endif
\end{verbatim}
To conditionally include {\tt foo.sml} when comiling under {\mlton}, one can
negate the test.
\begin{verbatim}
# if (! defined(SMLNJ_VERSION))
foo.sml
# endif
\end{verbatim}

The filenames listed in a CM file can be either absolute paths or relative
paths, in which case they are interpreted relative to the directory containing
the CM file.  If a CM file refers either directly or indirectly to an SML source
file in more than one way, only the first occurrence of the file is included.
Finally, the only valid file suffixes in a CM file are {\tt .cm}, {\tt .fun},
{\tt .sig}, and {\tt .sml}.

\subsec{Comparison with CM}{comparison}

If you are unfamiliar with CM under {\smlnj}, then you should skip this
section.

{\mlton} supports the full syntax of CM as of SML/NJ version 110.9.1.
Extensions since then are unsupported.  Also, many of the syntactic
constructs are ignored.  The most important difference between the two is that
order in CM files matters to {\mlton} but not to CM, which performs automatic
dependency analysis.  Also, CM supports export filters, which restricts the
visibility of modules.  {\mlton} ignores export filters.  As a consequence, it
is possible that a program that is accepted by SML/NJ's CM might not be accepted
by MLton's CM.  In this case, you will have to manually rename modules so that
the concatenation of the files is the program you intend.

CM performs cutoff recompilation to avoid recompiling the entire program, while
{\mlton} always compiles the entire program.  CM makes a distinction between
groups and libraries, which {\mlton} does not.  CM supports other tools like lex
and yacc, while {\mlton} does not.  {\mlton} relies on traditional makefiles to
use other tools.

\subsec{Porting {\smlnj} CM files to {\mlton}}{porting-cm}

If you have already created large projects using {\smlnj} and CM, there may be a
large number of file dependencies implicit in your sources that are not
reflected in your CM files.  Because {\mlton} relies on ordering in CM files,
your CM files probably will not work with {\mlton}.  To help in porting CM files
to {\mlton}, the {\mlton} distribution includes the SML sources for a
utility, {\tt cmcat}, that will print an ordered list of files corresponding to
a CM file.  See {\tt \doc/cmcat.sml} for details.  Building {\tt cmcat}
requires that you have already installed {\smlnj}.
