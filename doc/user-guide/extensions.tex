\sec{Modules specific to {\mlton}}{mlton}

This section describes the modules {\mlton} makes available that are
not part of the standard basis library.

\subsection{{\tt MLton}}

\begin{verbatim}
structure MLton:
   sig
      val debug: bool
      val isMLton: bool
      val safe: bool
      val size: 'a -> int
	 
      structure Cont: MLTON_CONT
      structure GC: MLTON_GC
      structure Itimer: MLTON_ITIMER
      structure ProcEnv: MLTON_PROC_ENV
      structure Random: MLTON_RANDOM
      structure Rlimit: MLTON_RLIMIT
      structure Rusage: MLTON_RUSAGE
      structure Signal: MLTON_SIGNAL
      structure Socket: MLTON_SOCKET
      structure Syslog: MLTON_SYSLOG
      structure Thread: MLTON_THREAD
      structure TextIO: MLTON_TEXT_IO
      structure Word: MLTON_WORD where type word = Word.word
      structure Word8: MLTON_WORD where type word = Word8.word
      structure World: MLTON_WORLD
   end
\end{verbatim}

\subsubsection{{\tt MLton}}

\newcommand{\entry}[1]{\item[\tt #1]\hspace{1in}\\}
\begin{description}

\entry{debug}
a compile-time constant that is true when compiling {\tt
-DMLton\_debug=1} and false when compiling {\tt -DMLton\_debug=0}.
The default is false.

\entry{isMLton} always true in a {\mlton} basis library.

\entry{\tt safe}
a compile time constant that makes it easy to
include assertions and checks in code that you would like removed when running
at full speed.  {\tt safe} will be true unless the program is compiled with the
{\tt -safe false} option, in which case it will be false.  The simplification
passes of the compiler will remove any uses of {\tt safe} at compile time.  For
example, array subscripting might be implemented as:
\begin{verbatim}
fun sub (a, i) =
   if MLton.safe andalso (i < 0 orelse i >= length a)
      then raise Subscript
   else unsafeSub (a, i)
\end{verbatim}
When compiled {\tt -safe false}, {\tt sub} will reduce to
{\tt unsafeSub}.

\entry{size x}
return the amount of heap space (in bytes) taken by the value of {\tt x}.  Size
takes time proportional to the size of its argument.  For an example, see {\tt
examples/size.sml}.

\end{description}

\subsubsec{{\tt MLton.Cont}}{cont}

\begin{verbatim}
signature MLTON_CONT =
   sig
      type 'a t

      val callcc: ('a t -> 'a) -> 'a
      val prepend: 'a t * ('b -> 'a) -> 'b t
      val throw: 'a t * 'a -> 'b
      val throw': 'a t * (unit -> 'a) -> 'b
   end
\end{verbatim}

\begin{description}

\entry{type 'a t}
the type of continuations that expect a value of type {\tt 'a}.

\entry{callcc f}
apply {\tt f} to the current continuation.  At present, {\tt
callcc} takes time proportional to the current stack size.

\entry{prepend (k, f)}
compose a function {\tt f} with a continuation {\tt k} to create a
continuation that first does {\tt f} and then does {\tt k}.  {\tt
prepend} is a constant time operation.

\entry{throw (k, v)}
throw value {\tt v} to continuation {\tt k}.  At present, {\tt throw}
takes time proportional to the size of {\tt k}.

\entry{throw' (k, th)}
a generalization of throw that evaluates {\tt th ()} in the context
of {\tt k}.  Thus, for example, if {\tt th ()} raises an exception or
grabs another continuation, it will see {\tt k}, not the current
continuation.

\end{description}

\subsubsection{{\tt MLton.GC}}
\begin{verbatim}
signature MLTON_GC =
   sig
      val collect: unit -> unit
      val setMessages: bool -> unit
      val setSummary: bool -> unit
   end
\end{verbatim}

\begin{description}
\entry{GC.collect ()}
causes a garbage collection to occur.

\entry{GC.setMessages}
controls whether diagnostic messages are
printed at the beginning and end of each garbage collection.  It is
the same as the {\tt gc-messages} runtime system option.

\entry{GC.setSummary}
controls whether a summary of garbage
collection statistics is printed upon termination of the program.  It
is the same as the {\tt gc-summary} runtime system option.

\end{description}

\subsubsec{{\tt MLton.Itimer}}{itimer}

\begin{verbatim}
signature MLTON_ITIMER =
   sig
      datatype which =
         Prof
       | Real
       | Virtual

      val set: which * {value: Time.time,
                        interval: Time.time} -> unit
      val whichSignal: which -> Posix.Signal.signal
   end
\end{verbatim}

\begin{description}
\entry{set (w, \{value, interval\})}
set the interval timer (using {\tt setitimer}) specified by {\tt w} to the given
{\tt value} and {\tt interval}.

\entry{whichSignal w}
return the signal corresponding to {\tt w}.
\end{description}

\subsubsection{\tt MLton.ProcEnv}
\begin{verbatim}
signature MLTON_PROC_ENV =
   sig
      val setenv: {name: string, value: string} -> unit
   end
\end{verbatim}

\begin{description}
\entry{setenv \{name, value\}}
Like the C {\tt setenv} function.
\end{description}

\subsubsection{\tt MLton.Random}
\begin{verbatim}
signature MLTON_RANDOM =
   sig
      val alphaNumString: int -> string
      val rand: unit -> word
      val seed: unit -> word
      val srand: word -> unit
      val useed: unit -> word
   end
\end{verbatim}

\begin{description}
\entry{alphaNumString n}
return a string of length {\tt n} of random alphanumeric characters.

\entry{rand ()}
return the next pseudrandom number.

\entry{seed ()}
return a random word from {\tt /dev/random}.  Useful as an arg to {\tt srand}.

\entry{srand w}
set the seed used by {\tt rand} to {\tt w}.

\entry{useed ()}
return a random word from {\tt /dev/urandom}.  Useful as an arg to {\tt srand}.
\end{description}

\subsubsection{\tt MLton.Rlimit}
This structure provides a wrapper around the C {\tt getrlimit} and {\tt
setrlimit}.
\begin{verbatim}
signature MLTON_RLIMIT =
   sig
      type rlim = word
	       
      val infinity: rlim

      type resource
	       
      val cpuTime: resource             (* CPU     CPU time in seconds *)
      val coreFileSize: resource        (* CORE    max core file size *)
      val dataSize: resource            (* DATA    max data size *)
      val fileSize: resource            (* FSIZE   Maximum filesize *)
      val lockedInMemorySize: resource  (* MEMLOCK max locked address space *)
      val numFiles: resource            (* NOFILE  max number of open files *)  
      val numProcesses: resource        (* NPROC   max number of processes *)
      val residentSetSize: resource     (* RSS     max resident set size *)
      val stackSize: resource           (* STACK   max stack size *)
      val virtualMemorySize: resource   (* AS      virtual memory limit *)
      
      val get: resource -> {hard: rlim, soft: rlim}
      val set: resource * {hard: rlim, soft: rlim} -> unit
   end
\end{verbatim}

\begin{description}
\entry{type rlim}
the type of resource limits.

\entry{infinity}
indicates that a resource is unlimited.

\entry{type resource}
type types of resources that can be inspected and modified.

\entry{get r}
returns the current hard and soft limits for resource {\tt r}.

\entry{set (r, \{hard, soft\})}
sets the hard and soft limits for resource {\tt r}.
\end{description}

\subsubsection{\tt MLton.Rusage}
\begin{verbatim}
signature MLTON_RUSAGE =
   sig
      type t = {utime: Time.time, (* user time *)
                stime: Time.time} (* system time *)
         
      val rusage: unit -> {children: t,
                           gc: t,
                           self: t}
   end
\end{verbatim}

\begin{description}
\entry{type t}
corresponds to a subset of the C {\tt struct rusage}.  We will add fields as
necessary.

\entry{rusage ()}
corresponds to the C {\tt getrusage} function.  It returns the resource usage of
the exited children, the garbage collector, and the process itself.  The process
time includes the gc time.
\end{description}

\subsubsec{{\tt MLton.Signal}}{signal}
Signals handlers are functions from threads to threads.  When a signal 
handler is invoked, it receives as an argument the thread that was
interrupted by the signal.  The signal handler returns the thread that 
it would like to resume execution.  It is an error for a signal
handler to raise an exception.  All signals are automatically blocked
for the duration of a handler.

\begin{verbatim}
signature MLTON_SIGNAL =
   sig
      include POSIX_SIGNAL

      val prof: signal
      val vtalrm: signal

      structure Mask:
         sig
            type t
               
            val all: t
            val some: signal list -> t
            val block: t -> unit
            val unblock: t -> unit
            val set: t -> unit
         end

      structure Handler:
         sig
            datatype t =
               Default
             | Ignore
             | Handler of unit Thread.t -> unit Thread.t

            val get: signal -> t
            val set: signal * t -> unit
            val simple: (unit -> unit) -> t
         end
   end
\end{verbatim}

\begin{description}

\entry{prof}
{\tt SIGPROF}, the profiling signal.

\entry{vtalrm}
{\tt SIGVTALRM}, the signal for virtual timers.

\entry{Mask.t} the type of signal masks

\entry{Mask.all} a mask of all signals.

\entry{Mask.some l} a mask of the signals in {\tt l}.

\entry{Mask.block m} block all signals in {\tt m}.

\entry{Mask.unblock m} unblock all signals in {\tt m}.

\entry{Mask.set m} set the signal mask to {\tt m}.

\entry{Handler.get s} return the current handler for signal {\tt s}.

\entry{Handler.set (s, h)} set the handler for signal {\tt s} to
be {\tt h}.

\entry{Handler.simple f} return a handler that executes {\tt f ()}
and then lets the interrupted thread continue.
\end{description}

\subsubsection{{\tt MLton.Socket}}
This module contains a bare minimum of functionality to do TCP/IP programming.
This module may disappear after the {\tt Socket} module of the standard basis
library becomes available.  Or, it may remain and be implemented on top of that
module. 
\begin{verbatim}
signature MLTON_SOCKET =
   sig
      structure Address:
         sig
            type t = word
         end
      structure Host:
         sig
            type t = {name: string}

            val getByAddress: Address.t -> t option
            val getByName: string -> t option
         end
      structure Port:
         sig
            type t = int
         end

      type t

      val accept: t -> Address.t * Port.t * TextIO.instream * TextIO.outstream
      val connect: string * Port.t -> TextIO.instream * TextIO.outstream
      val listen: unit -> Port.t * t
      val listenAt: Port.t -> t
      val shutdownRead: TextIO.instream -> unit
      val shutdownWrite: TextIO.outstream -> unit
   end
\end{verbatim}

\begin{description}

\entry{type Address.t}
the type of IP addresses.

\entry{Host.getByAddress a}
lookup the hostname (using {\tt gethostbyaddr}) corresponding to {\tt a}.

\entry{Host.getByName s}
lookup the hostname (using {\tt gethostbyname}) corresponding to {\tt s}.

\entry{type Port.t}
the type of TCP ports.

\entry{type t}
the type of sockets.

\entry{accept s}
accept a connection on socket {\tt s} and return the address and
port of the connecting socket, as well as streams corresponding to the
connection.

\entry{connect (h, p)}
connect to host {\tt h} on port {\tt p}, returning the streams
corresponding to the connection.

\entry{listen ()}
listen to a port chosen by the system.  Returns the port and the socket.

\entry{listenAt p}
listen to port {\tt p}.  Returns the socket.

\entry{shutdownRead ins}
cause the read part of the socket associated with {\tt ins} to be shutdown.

\entry{shutdownWrite out}
cause the write part of the socket associated with {\tt out} to be shutdown.
\end{description}

\subsubsection{\tt MLton.Syslog}
A complete interface to the system logging facilities.  See {\tt man 3 syslog}
for more details.
\begin{verbatim}
signature MLTON_SYSLOG =
   sig
      type openflag
       val CONS     : openflag
       val NDELAY   : openflag
       val PERROR   : openflag
       val PID      : openflag
   
       type facility
       val AUTHPRIV : facility
       val CRON     : facility
       val DAEMON   : facility
       val KERN     : facility
       val LOCAL0   : facility
       val LOCAL1   : facility
       val LOCAL2   : facility
       val LOCAL3   : facility
       val LOCAL4   : facility
       val LOCAL5   : facility
       val LOCAL6   : facility
       val LOCAL7   : facility
       val LPR      : facility
       val MAIL     : facility
       val NEWS     : facility
       val SYSLOG   : facility
       val USER     : facility
       val UUCP     : facility
   
       type loglevel
       val EMERG    : loglevel
       val ALERT    : loglevel
       val CRIT     : loglevel
       val ERR      : loglevel
       val WARNING  : loglevel
       val NOTICE   : loglevel
       val INFO     : loglevel
       val DEBUG    : loglevel
   
       val closelog: unit -> unit
       val log: loglevel * string -> unit
       val openlog: string * openflag list * facility -> unit
   end
\end{verbatim}

\begin{description}
\entry{closelog ()}
close the connection to the system logger.

\entry{log (l, s)}
log message {\tt s} at a loglevel {\tt l}.

\entry{openlog (name, flags, facility)}
open a connection to the system logger. {\tt name} will be prefixed to each
message, and is typically set to the program name.

\end{description}

\subsubsection{{\tt MLton.TextIO}}
\begin{verbatim}
signature MLTON_TEXT_IO =
   sig
      val inFd: TextIO.instream -> Posix.IO.file_desc
      val mkstemp: string -> string * TextIO.outstream
      val mkstemps: {prefix: string, suffix: string} -> string * TextIO.outstream
      val newIn: Posix.IO.file_desc -> TextIO.instream
      val newOut: Posix.IO.file_desc -> TextIO.outstream
      val outFd: TextIO.outstream -> Posix.IO.file_desc
   end
\end{verbatim}

\begin{description}

\entry{inFd ins} return the file descriptor corresponding to
{\tt ins}.

\entry{mkstemp s}
like the C {\tt mkstemp} function, generate and open a tempory file with prefix
{\tt s}.  This should be used instead of {\tt OS.FileSys.tmpName}, which has
security risks.

\entry{mkstemps \{prefix, suffix\}}
{\tt mkstemps} is like {\tt mkstemp}, except it has both a prefix and suffix.

\entry{newIn fd} create a new instream from file descriptor {\tt 
fd}.

\entry{newOut} create a new outstream from file descriptor {\tt 
fd}.

\entry{outFd out} return the file descriptor corresponding to
{\tt out}.
\end{description}

\subsubsec{{\tt MLton.Thread}}{thread}
Threads are data structures that will begin computing when {\tt
switch}ed to with a value.
\begin{verbatim}
signature MLTON_THREAD =
   sig
      type 'a t

      val atomicBegin: unit -> unit
      val atomicEnd: unit -> unit
      val new: ('a -> unit) -> 'a t
      val prepend: 'a t * ('b -> 'a) -> 'b t
      val switch: ('a t -> 'b t * 'b) -> 'a
      val switch': ('a t -> 'b t * (unit -> 'b)) -> 'a
   end
\end{verbatim}

\begin{description}

\entry{type 'a t}
the type of threads that expect a value of type {\tt 'a}.

\entry{atomicBegin ()}
begin a critical section.

\entry{atomicEnd ()}
end a critical section.

\entry{new f}
create a new thread that, when run, applies {\tt f} to the
value given to the thread. 

\entry{prepend (t, f)}
create a new thread (destroying {\tt t} in the process) that first applies
{\tt f} to the value given to the thread and then continues with {\tt
t}.  This is a constant time operation.

\entry{switch f}
apply {\tt f} to the current thread to get
{\tt (t, v)}, and then start running thread {\tt t} on value {\tt
v}.  It is an error for {\tt f} to perform another {\tt switch}.  Also, {\tt f}
is guaranteed to run atomically.

\entry{switch' f}
apply {\tt f} to the current thread to get
{\tt (t, g)}, and then starts running thread {\tt t} computing the
value {\tt g ()}.

\end{description}

\subsubsection{\tt MLton.Word, MLton.Word8}
\begin{verbatim}
signature MLTON_WORD =
   sig
      type word
	 
      val ~ : word -> word
      val rol: word * Word.word -> word
      val ror: word * Word.word -> word
   end
\end{verbatim}

\begin{description}
\entry{\~{} w}
bitwise complement of {\tt w}.
\entr{rol (w, w')}
rotate left.
\entr{ror (w, w')}
rotate right.
\end{description}

\subsubsec{{\tt MLton.World}}{world}
\begin{verbatim}
signature MLTON_WORLD =
   sig
      datatype status = Clone | Original

      val load: string -> 'a
      val save: string -> status
      val saveThread: string * unit Thread.t -> status
   end
\end{verbatim}

\begin{description}
\entry{datatype status}
used to specify whether a world is original or a restarted (a clone).

\entry{load f}
load the saved computation from file f.

\entry{save f}
save the entire state of the computation to
the file {\tt f}.  The computation can then be restarted at a later
time using {\tt World.load} or the {\tt load-world} runtime system
option.  The call to {\tt save} in the original computation returns
{\tt Original} and the call in the restarted world returns {\tt
Clone}.  The following example is a transcript run in the {\tt
examples/save-world} directory.
\begin{verbatim}
% mlton save-world.sml
% save-world
I am the original
% save-world @MLton load-world world --
I am the clone
\end{verbatim}

\entry{saveThread (f, t)}
save the entire state of the computation to
the file {\tt f}, and resume with thread {\tt t} upon restart.

\end{description}

\subsection{{\tt SMLofNJ: SML\_OF\_NJ}}

{\tt SMLofNJ} implements a subset of the structure of the same name
provided in Standard ML of New Jersey.  It is included to make it
easier to port programs between the two systems.  The semantics of
these functions may be different than in {\smlnj}.

\begin{verbatim}
signature SML_OF_NJ =
   sig
      structure Cont:
         sig
            type 'a cont
            val callcc: ('a cont -> 'a) -> 'a
            val throw: 'a cont -> 'a -> 'b
         end
      structure SysInfo:
         sig
	    exception UNKNOWN
	    datatype os_kind = BEOS | MACOS | OS2 | UNIX | WIN32

	    val getHostArch: unit -> string
	    val getOSKind: unit -> os_kind
	    val getOSName: unit -> string
         end
      structure Internals:
         sig
            structure GC:
               sig
                  val messages: bool -> unit
               end             
         end

      val exportFn: string * (string * string list -> OS.Process.status) -> unit
      val exportML: string -> bool
      val getCmdName: unit -> string
      val getArgs: unit -> string list
      val getAllArgs: unit -> string list
   end
\end{verbatim}

\begin{description}

\entry{structure Cont} implements continuations.

\entry{SysInfo.getHostArch ()}
returns {\tt "X86"}.

\entry{SysInfo.getOSKind}
returns {\tt UNIX}.

\entry{SysInfo.getOSName ()}
returns {\tt "Linux"}.

\entry{Internals.GC.messages b}
the same as {\tt MLton.GC.messages b}.

\entry{getCmdName ()}
the same as {\tt CommandLine.name ()}.

\entry{getArgs ()}
the same as {\tt CommandLine.arguments ()}.

\entry{getAllArgs ()}
the same as {\tt getCmdName() :: getArgs()}.

\entry{exportFn f}
save the state of the computation to a file that will apply {\tt f} to
the command-line arguments upon restart.

\entry{exportML f}
save the state of the computation to file {\tt f} and continue.
Return {\tt true} in the restarted computation and {\tt false} in the
continuing computation.

\end{description}

\subsection{{\tt Unsafe: UNSAFE}}

This module is a subset of the {\tt Unsafe} module provided by
{\smlnj}.  It is included in {\mlton} because the code generated by
ML-Yacc includes references to unsafe subscript operations.

\begin{verbatim}
signature UNSAFE =
  sig
    structure Vector: UNSAFE_VECTOR
    structure Array: UNSAFE_ARRAY

    structure CharVector: UNSAFE_MONO_VECTOR
      where type vector = CharVector.vector
      where type elem = CharVector.elem
    structure CharArray: UNSAFE_MONO_ARRAY
      where type array = CharArray.array
      where type elem = CharArray.elem

    structure Word8Vector: UNSAFE_MONO_VECTOR
      where type vector = Word8Vector.vector
      where type elem = Word8Vector.elem
    structure Word8Array: UNSAFE_MONO_ARRAY
      where type array = Word8Array.array
      where type elem = Word8Array.elem
    structure Real64Array: UNSAFE_MONO_ARRAY
      where type array = Real64Array.array
      where type elem = Real64Array.elem
  end
\end{verbatim}
