\sec{Getting started}{getting-started}

\subsection{Installation}

{\mlton} runs on X86 machines under Linux.  {\mlton} is distributed in source
and binary RPM and gzipped tar format.  To install the binary RPM, run the
following command.
\begin{alltt}
rpm -i mlton-\version-1.i386.rpm
\end{alltt}
The binary RPM installs the following files and directories.

\newcommand{\place}[1]{\item[\tt #1]\hspace{1in}\\}

\begin{description}

\place{/usr/local/bin/mllex}
A lexer generator.

\place{/usr/local/bin/mlprof}
A profiler (see \secref{profiling}).

\place{/usr/local/bin/mlton}
A script to call the compiler.
This script may be moved anywhere, however,
it makes use of files in {\tt /usr/local/lib/mlton}.

\place{/usr/local/bin/mlyacc}
A parser generator.

\place{/usr/local/man/man1/mlprof.1}
Man page for {\tt mlprof}.

\place{/usr/local/man/man1/mlton.1}
Man page for {\tt mlton}.

\place{\doc}
Directory containing the user guide, example SML programs (in the {\tt examples}
dir), and license information.

\place{/usr/local/lib/mlton}
Directory containing libraries and include files needed during
compilation.

\end{description}

\subsection{Hello, World!}

Create a file called {\tt hello-world.sml} with the following contents.

\begin{verbatim}
print "Hello, world!;\n"
\end{verbatim}
Now, create an executable called {\tt hello-world} with the following command.
\begin{verbatim}
mlton hello-world.sml
\end{verbatim}
You can now run this executable to verify that it works.  There are several
other small examples in {\tt examples}.  In particular, there are examples that
demonstrate world save and restore, callcc, object size primitive, threads,
profiling, and the FFI.
