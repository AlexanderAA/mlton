\sec{Getting started}{getting-started}

\subsection{Installation}

{\mlton} runs on X86 machines with either Linux, FreeBSD, or Cygwin/Windows.
and is distributed in source and binary RPM and gzipped tar ({\tt
tgz}) format.  You can install the binary RPM with\\
\hspace*{2em}{\tt rpm -i mlton-\version-1.i386.rpm}\\
You can install the binary tgz with\\
\hspace*{2em}{\tt zcat mlton-\version-1.i386-linux.tgz | tar x}\\
Or, if your distribution doesn't use rpm, but you have {\tt rpm2cpio},
you can use\\
\hspace*{2em}{\tt rpm2cpio mlton-\version-1.i386.rpm | cpio -id}\\
If you do not install {\mlton} in the root directory, you must
set the {\tt lib} variable in {\tt \prefix/bin/mlton} to the
directory that contains the libraries ({\tt /\prefix/lib/mlton} by
default).

{\mlton} requires that you have the
\htmladdnormallink{GNU multiprecision library (gmp)}
		  {http://www.gnu.org/software/gmp/gmp.html}
installed on your machine.  {\mlton} must be able to find both the
{\tt gmp.h} include and the {\tt libgmp.a} or {\tt libgmp.so} library.
If you see the error message {\tt gmp.h: No such file or directory},
you should copy {\tt gmp.h} to {\tt usr/lib/mlton/include/self}.  If
you see the error message, {\tt /usr/bin/ld: cannot find -lgmp}, you
should add a {\tt -L} argument in the {\tt usr/bin/mlton} script so
that the linker can find {\tt libgmp}, for example, {\tt
-L/\prefix/lib}.

Installation of {\mlton} creates the following files and directories.

\newcommand{\place}[1]{\item[\tt #1]\hspace{1in}\\}

\begin{description}

\place{\prefix/bin/mllex}
A lexer generator.

\place{\prefix/bin/mlprof}
A profiler (see \secref{profiling}).

\place{\prefix/bin/mlton}
A script to call the compiler.
This script may be moved anywhere, however,
it makes use of files in {\tt /\prefix/lib/mlton}.

\place{\prefix/bin/mlyacc}
A parser generator.

\place{\prefix/share/man/man1/mlprof.1}
Man page for {\tt mlprof}.

\place{\prefix/share/man/man1/mlton.1}
Man page for {\tt mlton}.

\place{\prefix/share/doc/mlton}

Directory containing the user guide, example SML programs (in the {\tt examples}
dir), and license information.

\place{\prefix/lib/mlton}
Directory containing libraries and include files needed during
compilation.

\end{description}

\subsection{Hello, World!}

Once you have installed {\mlton}, create a file called {\tt
hello-world.sml} with the following contents.

\begin{verbatim}
print "Hello, world!\n";
\end{verbatim}
Now, create an executable, {\tt hello-world}, with the following command.
\begin{verbatim}
mlton hello-world.sml
\end{verbatim}
You can now run this executable to verify that it works.  There are
more small examples in {\tt \prefix/share/doc/mlton/examples}.  In
particular, there are examples that demonstrate world save and
restore, callcc, object size primitive, threads, profiling, and the
FFI.
