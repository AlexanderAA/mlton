\sec{Running on FreeBSD}{freebsd}

We have noticed a few issues when running {\mlton} on FreeBSD.  

\begin{itemize}

\item The executables often run more slowly than on a comparable Linux machine.
We conjecture that part of this is due to costs due to heap resizing and kernel
zeroing of pages.

\end{itemize}

We have also noticed a few issues when compiling {\mlton} on FreeBSD.  These
only arise if you are working with the {\mlton} sources.

\begin{itemize}

\item {\mlton} {\tt Makefile}s use GNU extensions, so you should use {\tt gmake}
instead of {\tt make}.

\item {\mlton} requires GNU MP version 3 or higher.  The default that comes with
FreeBSD is version 2.  So to build MLton, in particular the runtime, requires
putting gmp.h and libgmp.a in the {\tt runtime} directory.  You can get them
by either downloading from \htmlurl{http://www.gnu.org/software/gmp/gmp.html}
and compiling yourself (we use 4.0.1), or just copying them from a {\mlton}
binary package for FreeBSD ({\tt /usr/local/lib/mlton/self/libgmp.a} or {\tt
/usr/local/lib/mlton/self/include/gmp.h}).

By default, {\tt runtime/Makefile} will grab them from the {\mlton} binary
package.

\end{itemize}
