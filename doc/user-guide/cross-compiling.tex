\sec{Cross compiling applications for Windows via Cygwin}{cross-compiling}

Although {\mlton} runs on Linux, you can use the {\tt -host} flag to cross
compile applications and produce executables that run on Windows.  {\mlton} uses
the \htmladdnormallink{Cygwin}{http://www.cygwin.com/} Unix emulation layer for
Windows to provide a Posix-like environment.  In order to use {\mlton} as a
cross compiler, you need to do several things.

\begin{enumerate}

\item Install the Cygwin dll in the Windows machine.

\item Install the GCC cross-compiler tools on your Linux machine.

\item Cross compile the {\mlton} runtime system for your Windows machine.

\end{enumerate}

To install the Cygwin dll, visit the Cygwin site from your Windows machine and
run their {\tt setup.exe} script.

To build a GCC cross-compiler toolset on your machine, you can use the script
{\tt bin/build-cross-gcc} available in the {\mlton} sources.  There are some
comments at the top of the script that tell you what to download and what
variables to set in order to build the toolset.  In particular, the {\tt target}
variable is important, since that is what you will pass to {\mlton}'s {\tt
-host} flag.

Once you have the toolset built, you should be able to test it by cross
compiling a simple hello world program on your Linux machine.
\begin{verbatim}
gcc -b i386-pc-cygwin -o hello-world.exe hello-world.c
\end{verbatim}
You should now be able to run {\tt hello-world.exe} from a Cygwin shell on your
Windows Machine.

Next, you must cross compile the {\mlton} runtime system and inform {\mlton} of
the availability of the new target.  The script {\tt bin/add-cross} from
the {\mlton} sources that will help you do this.  Please read the comments at
the top of the script.  Here is a sample run.
\begin{verbatim}
% add-cross
Making runtime.
Building print-constants executable.
You must now run print-constants.exe on the i386-pc-cygwin machine
and put the output in /home/sweeks/mlton/src/build/lib/i386-pc-cygwin/constants.
\end{verbatim}
Running {\tt add-cross} installs the cross-compiled runtime and creates a
cross-compiled executable, {\tt print-constants.exe}, which prints out all of
the constants that {\mlton} needs in order to implement the basis library.  The
final step is to run {\tt print-constants.exe} on your Windows machine, and save
the output in the file indicated by {\tt add-cross}.  Once you have done this,
you should be able to cross compile SML applications.  For example
\begin{verbatim}
mlton -host i386-pc-cygwin hello-world.sml
\end{verbatim}
will create {\tt hello-world.exe}, which you should be able to run from a Cygwin
shell on your Windows machine.

\subsec{Known problems}{known-problems}

There are several known problems with cross-compiled SML applications, which are
due to bugs in Cygwin's emulation.

\begin{enumerate}

\item Due to a bug in mmap/fork, fork does not work correctly.  Any call to {\tt
Posix.Process.fork} will raise the Fail exception.  The Cygwin developers claim
to have fixed the bug (3/11/02) and I will try it out soon.  This problem may be
avoided in some situations by using spawn, which I will add to the MLton basis
library soon.

\item Due to a bug in mmap, large heaps do not work correctly.  This may
cause seg faults on programs that use a lot of memory.  I am working with the
Cygwin developers to fix the problem.

\end{enumerate}
