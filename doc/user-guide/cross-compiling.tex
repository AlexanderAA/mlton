\sec{Cross compiling}{cross-compiling}

You can use the {\mlton}'s {\tt -host} flag to cross compile
applications.  By default, {\mlton} is only able to compile for the
machine it is running on.  In order to use {\mlton} as a cross
compiler, you need to do two things.  To make the terminology clear,
we refer to the the {\em host} as the machine {\mlton} is running on
and the {\em target} as the machine that {\mlton} is compiling for.

\begin{enumerate}

\item Install the GCC cross-compiler tools on the host so that GCC can
compile to the target.

\item Cross compile the {\mlton} runtime system to build the runtime
libraries for the target.

\end{enumerate}

To build a GCC cross-compiler toolset on the host, you can use the
script {\tt bin/build-cross-gcc}, available in the {\mlton} sources,
as a template.  The value of the {\tt target} variable in that script
is important, since that is what you will pass to {\mlton}'s {\tt
-host} flag.

Once you have the toolset built, you should be able to test it by
cross compiling a simple hello world program on your host machine.
\begin{verbatim}
gcc -b i386-pc-cygwin -o hello-world hello-world.c
\end{verbatim}
You should now be able to run {\tt hello-world} on the target machine,
in this case, a Cygwin machine.

Next, you must cross compile the {\mlton} runtime system and inform
{\mlton} of the availability of the new target.  The script {\tt
bin/add-cross} from the {\mlton} sources will help you do this.
Please read the comments at the top of the script.  Here is a sample
run adding a SunOS cross compiler.
\begin{verbatim}
% add-cross sparc-sun-solaris sun blade
Making runtime.
Building print-constants executable.
Running print-constants on blade.
\end{verbatim}
Running {\tt add-cross} installs the cross-compiled runtime and
creates a cross-compiled executable, {\tt print-constants}, which
prints out all of the constants that {\mlton} needs in order to
implement the basis library.  Then, it runs {\tt print-constants} on
the target machine ({\tt blade} in this case), and saves the output.

Once you have done all this, you should be able to cross compile SML
applications.  For example,
\begin{verbatim}
mlton -host i386-pc-cygwin hello-world.sml
\end{verbatim}
will create {\tt hello-world}, which you should be able to run from a
Cygwin shell on your Windows machine.
