\sec{Cross compiling applications for Windows via Cygwin}{cross-compiling}

Although {\mlton} runs on Linux, you can use the {\tt -host} flag to cross
compile applications and produce executables that run on Windows.  {\mlton} uses
the \htmladdnormallink{Cygwin}{http://www.cygwin.com/} Unix emulation layer for
Windows to provide a Posix-like environment.  In order to use {\mlton} as a
cross compiler, you need to do several things.

\begin{enumerate}

\item Install the Cygwin dll in the Windows machine.

\item Install the GCC cross-compiler tools on your Linux machine.

\item Cross compile the {\mlton} runtime system for your Windows machine.

\end{enumerate}

To install the Cygwin dll, visit the Cygwin site from your Windows machine and
run their {\tt setup.exe} script.

To build a GCC cross-compiler toolset on your machine, you can use the script
{\tt bin/build-cross-gcc} available in the {\mlton} sources.  There are some
comments at the top of the script that tell you what to download and what
variables to set in order to build the toolset.  In particular, the {\tt target}
variable is important, since that is what you will pass to {\mlton}'s {\tt
-host} flag.

Once you have the toolset built, you should be able to test it by cross
compiling a simple hello world program on your Linux machine.
\begin{verbatim}
gcc -b i386-pc-cygwin -o hello-world.exe hello-world.c
\end{verbatim}
You should now be able to run {\tt hello-world.exe} from a Cygwin shell on your
Windows Machine.

Next, you must cross compile the {\mlton} runtime system and inform {\mlton} of
the availability of the new target.  The script {\tt bin/add-cross} from
the {\mlton} sources that will help you do this.  Please read the comments at
the top of the script.  Here is a sample run.
\begin{verbatim}
% add-cross
Making runtime.
Building print-constants executable.
You must now run print-constants.exe on the i386-pc-cygwin machine
and put the output in /home/sweeks/mlton/src/build/lib/i386-pc-cygwin/constants.
\end{verbatim}
Running {\tt add-cross} installs the cross-compiled runtime and creates a
cross-compiled executable, {\tt print-constants.exe}, which prints out all of
the constants that {\mlton} needs in order to implement the basis library.  The
final step is to run {\tt print-constants.exe} on your Windows machine, and save
the output in the file indicated by {\tt add-cross}.  Once you have done this,
you should be able to cross compile SML applications.  For example
\begin{verbatim}
mlton -host i386-pc-cygwin hello-world.sml
\end{verbatim}
will create {\tt hello-world.exe}, which you should be able to run from a Cygwin
shell on your Windows machine.

\subsec{Known problems}{known-problems}

There are several known problems with cross-compiled SML applications, which are
due to bugs in Cygwin's emulation.

\begin{enumerate}

\item There is a bug in fork, which I am still investigating.  For idiomatic
uses of fork + exec, this problem may be avoided by using the {\tt
MLton.Process.spawn} family of functions. 

\item There was a bug in mmap that caused large heaps to not work correctly.
This may even cause seg faults on programs that use a lot of memory.  There was
a fix available in the Cygwin developer snapshot 2002-Mar-16 available at
{\tt http://cygwin.com/snapshots/}.  I am still unsure if problems remain.

\item I am not confident in the correctness of Cygwin's emulation of signals and
signal handlers, and am still investigating them.

\end{enumerate}
