\sec{ML Basis system}{mlb}{MLBs.html}
%
This section describes the ML Basis system (MLBs) used in {\mlton}.
While the Modules level of Standard ML provides a sophisticated
language for programming-in-the-large, it is difficult, if not
impossible, to accomplish a number of routine namespace management
operations that are necessary for
programming-in-the-\emph{very}-large, when (parts of) a program draws
upon multiple libraries provided by different vendors.  The ML Basis
system is a simple, yet powerful, approach that builds upon the
programmer's intuitive notion (and the Definition of SML's formal
notion) of the top-level environment (a \emph{basis}).  The system has
been designed to be a natural extension of Standard ML;
the
\htmladdnormallink{Formal Specification of MLBs}
                  {http://www.mlton.org/doc/MLBs/mlb-formal.ps}
is given in the style of the Definition.

We briefly highlight some of the key features provided by MLBs:
\begin{itemize}
\item Explicit file order: The order of files (and, hence, the order
  of evaluation) in the program is explicit.  The MLBs semantics are
  structured in such a way that for any well-formed project, there
  will be exactly one possible interpretation of the project's syntax,
  static semantics, and dynamic semantics.
\item Implicit dependencies: A source file (corresponding to a SML
  top-level declaration) is elaborated in the environment described by
  preceding declarations.  It is not necessary to explicitly list the
  dependencies of a file.
\item Scoping and renaming: MLBs provides mechanisms for limiting the
  scope of (i.e, hiding) and renaming identifiers.
\item No file name convention: MLBs assumes no convention for finding
  the file that defines a module.  (Conversely, to import a module,
  it's defining file must appear in the project.)
\end{itemize}

%
\subsection{Syntax and semantics}

An {\tt .mlb} (ML Basis) file describes a library or program.  An
{\tt .mlb} file contains a ``basis declaration,'' defined by the
following grammar:

\begin{latexonly}
\begin{center}
\begin{tabular}{lcl}
{\it basdec} 
& ::=    & {\tt basis} {\it basid} {\tt =} {\it basexp}
            ({\tt and} {\it basid} {\tt =} {\it basexp})* \\
& $\alt$ & {\tt open} {\it basid}$_1$ $\cdots$ {\it basid}$_n$ \\
& $\alt$ & {\tt local} {\it basdec} {\tt in} {\it basdec} {\tt end} \\
& $\alt$ & {\it basdec} [{\tt;}] {\it basdec} \\
& $\alt$ & {\tt structure} {\it strid} [{\tt =} {\it strid}] 
                ({\tt and} {\it strid} [{\tt =} {\it strid}])* \\
& $\alt$ & {\tt signature} {\it sigid} [{\tt =} {\it sigid}] 
                ({\tt and} {\it sigid} [{\tt =} {\it sigid}])* \\
& $\alt$ &   {\tt functor} {\it funid} [{\tt =} {\it funid}] 
                ({\tt and} {\it funid} [{\tt =} {\it funid}])* \\
& $\alt$ & {\it path.}{\tt sml} \\
& $\alt$ & {\it path.}{\tt mlb} \\
& $\alt$ & {\tt ann} {\it id} ({\it id})* ({\tt ,}  {\it id} ({\it id})*)*
           {\tt in} {\it basdec} {\tt end} \\
\\
{\it basexp}
& ::=    & {\tt bas} {\it basdec} {\tt end} \\
& $\alt$ & {\it basid} \\
& $\alt$ & {\tt let} {\it basdec} {\tt in} {\it basexp} {\tt end}
\end{tabular}
\end{center}
\end{latexonly}
\begin{htmlonly}
\begin{center}
\begin{tabular}{lcl}
{\it basdec} 
& ::=    & {\tt basis} {\it basid} {\tt =} {\it basexp}
            ({\tt and} {\it basid} {\tt =} {\it basexp})* \\
& | & {\tt open} {\it basid}$_1$ $\cdots$ {\it basid}$_n$ \\
& | & {\tt local} {\it basdec} {\tt in} {\it basdec} {\tt end} \\
& | & {\it basdec} [{\tt;}] {\it basdec} \\
& | & {\tt structure} {\it strid} [{\tt =} {\it strid}] 
                ({\tt and} {\it strid} [{\tt =} {\it strid}])* \\
& | & {\tt signature} {\it sigid} [{\tt =} {\it sigid}] 
                ({\tt and} {\it sigid} [{\tt =} {\it sigid}])* \\
& | &   {\tt functor} {\it funid} [{\tt =} {\it funid}] 
                ({\tt and} {\it funid} [{\tt =} {\it funid}])* \\
& | & {\it path.}{\tt sml} \\
& | & {\it path.}{\tt mlb} \\
& | & {\tt ann} {\it id} ({\it id})* ({\tt ,}  {\it id} ({\it id})*)*
           {\tt in} {\it basdec} {\tt end} \\
\\
{\it basexp}
& ::=    & {\tt bas} {\it basdec} {\tt end} \\
& | & {\it basid} \\
& | & {\tt let} {\it basdec} {\tt in} {\it basexp} {\tt end}
\end{tabular}
\end{center}
\end{htmlonly}

Nested SML-style comments (enclosed with {\tt (*} and {\tt *)}) are
ignored (but {\nline} directives of \secref{nline} are recognized).

Conceptually, a basis file is elaborated starting in an empty basis,
and each basis declaration produces a basis as a result.  Basis
expressions and basis identifiers allow binding a basis to a name;
this, in turn, allows fine-grained specification of dependenies,
without the need for additional {\tt .mlb} files.  Local declarations
provide name hiding.  Sequencing of basis declarations merges the
bases.  Structure, signature, and functor declarations bind a module
in the current basis.

References to SML source files cause the file to be elaborated in the
``current'' basis.  References to other ML basis files cause the basis
denoted by that ML basis file to be imported.  Recall that an ML basis
file is elaborated in an empty basis; hence, no bindings from the
``current'' basis are available to the imported basis file.  Since
{\tt .mlb} files are elaborated in the empty basis, they need only be
elaborated (and evaluated) once.  The semantics of MLBs are such that
the results of elaborating (and evaluating) a {\tt .mlb} file are
cached.  Thus, any observable effects due to evaluation are not
duplicated if the {\tt .mlb} file is referred to multiple times.

Paths can be relative or absolute.  Relative paths are relative to the
directory containing the {\tt .mlb} file.  Paths may include path
variables and are expanded according to a path map; see
\secref{mlb-path-map} for more details.  Unquoted paths may include
alpha-numeric characters and the symbols {\tt -} and {\tt \_}, along
with the arc separator {\tt /} and extension separator {\tt .}.  More
complicated paths, including paths with spaces, may be included by
quoting the path with {\tt "}.  A quoted path is lexed as a SML string
constant.

Finally, annotations allow a library author to control options that
affect the elaboration of SML source files; see \secref{mlb-anns} for
more details.

%
\subsec{Examples}{mlb-examples}

We demonstrate how to accomplish some common tasks:
\begin{description}
\item{Complete program:}
Suppose your complete program consists of the files {\tt file1.sml},
$\ldots$, {\tt filen.sml}, which depend upon libraries {\tt lib1.mlb},
$\ldots$, {\tt libm.mlb}.
\begin{verbatim}
(* import libraries *)
lib1.mlb
...
libm.mlb

(* program files *)
file1.sml
...
filen.sml
\end{verbatim}
The bases denoted by {\tt lib1.mlb}, $\ldots$, {\tt libm.mlb} are
merged (bindings of names in later bases take precedence over bindings
of the same name in earlier bases), producing a basis in which {\tt
file1.sml}, $\ldots$, {\tt filen.sml} are elaborated, possibly adding
additional bindings to the basis.
\item{Export filter:}
Suppose you only want to export certain structures, signatures, and
functors from a collection of files.
\begin{verbatim}
local
  file1.sml
  ...
  filen.sml
in
  (* export filter here *)
  functor F
  structure S
end
\end{verbatim}
While {\tt file1.sml}, $\ldots$, {\tt filen.sml} may declare top-level
identifiers in addition to {\tt F} and {\tt S}, such names are not
accessible to programs and libraries that import this {\tt .mlb}.
\item{Export filter with renaming:}
Suppose you want an export filter, but want to rename one of the
modules.
\begin{verbatim}
local
  file1.sml
  ...
  filen.sml
in
  (* export filter, with renaming, here *)
  functor F
  structure S' = S
end
\end{verbatim}
Note that {\tt functor F} is an abbreviation for {\tt functor F = F},
which simply exports an identifier under the same name.
\item{Import filter:}
Suppose you only want to import a functor {\tt F} from one library and
a structure {\tt S} from another library.
\begin{verbatim}
local
  lib1.mlb
in
  (* import filter here *)
  functor F
end
local
  lib2.mlb
in
  (* import filter here *)
  structure S
end
file1.sml
...
filen.sml
\end{verbatim}
\item{Import filter with renaming:}
Suppose you want to import a structure {\tt S} from one library and another
structure {\tt S} from another library.
\begin{verbatim}
local
  lib1.mlb
in
  (* import filter, with renaming, here *)
  structure S1 = S
end
local
  lib2.mlb
in
  (* import filter, with renaming, here *)
  structure S2 = S
end
file1.sml
...
filen.sml
\end{verbatim}
\end{description}

Since the Modules level of SML is the natural means for organizing
program and library components, MLBs provides convenient syntax for
renaming Modules level identifiers.\footnote{In fact, renaming of
functor identifiers provides a mechanism that is not available in
SML.}  However, please note that {\tt .mlb} files elaborate to full
bases including top-level types and values (including infix status),
in addition to structures, signatures, and functors.  For example,
suppose you wished to extend the Standard ML Basis Library with an
{\tt ('a, 'b) either} datatype corresponding to a disjoint sum; the
type and some operations should be available at the top-level;
additionally, a signature and structure provide the complete
interface.

We assume that the main implementation is given by two files: {\tt
either-sigs.sml} and {\tt either-strs.sml}:

\begin{description}
\item{{\tt either-sigs.sml}:}
\begin{verbatim}
signature EITHER_GLOBAL =
  sig
    datatype ('a, 'b) either = Left of 'a | Right of 'b
    val &  : ('a -> 'c) * ('b -> 'c) -> ('a, 'b) either -> 'c
    val && : ('a -> 'c) * ('b -> 'd) -> ('a, 'b) either -> ('c, 'd) either
  end

signature EITHER =
  sig
    include EITHER_GLOBAL
    val isLeft  : ('a, 'b) either -> bool
    val isRight : ('a, 'b) either -> bool
    ...
  end
\end{verbatim}
\item{{\tt either-strs.sml}:}
\begin{verbatim}
structure Either : EITHER =
  struct
    datatype ('a, 'b) either = Left of 'a | Right of 'b
    fun f & g = fn x =>
      case x of Left z => f z | Right z => g z
    fun f && g = fn x =>
      ((Left o f) & (Right o g)) x
    fun isLeft x = ((fn _ => true) & (fn _ => false)) x
    fun isRight x = (not o isLeft) x
    ...
  end
structure EitherGlobal : EITHER_GLOBAL = Either
\end{verbatim}
\end{description}

Two additional files contain the infix directives ({\tt
either-infixes.sml}) and a declaration to import the top-level types
and values ({\tt either-open.sml}):
\begin{description}
\item{{\tt either-infixes.sml}:}
\begin{verbatim}
infixr 3 & &&
\end{verbatim}
\item{{\tt either-open.sml}:}
\begin{verbatim}
open EitherGlobal
\end{verbatim}
\end{description}

The extension is delivered via {\tt either.mlb}:
\begin{description}
\item{{\tt either.mlb}:}
\begin{verbatim}
local
  (* import Basis Library *)
  basis.mlb

  either-sigs.sml
  either-infixes.sml
  either-strs.sml
in
  signature EITHER
  structure Either
  either-infixes.sml
  either-open.sml
end
\end{verbatim}
\end{description}
A client that imports {\tt either.mlb} will have access to neither
{\tt EITHER\_GLOBAL} nor {\tt EitherGlobal}, but will have access to
the type {\tt either} and the values {\tt \&} and {\tt \&\&} (with
infix status) in the top-level environment.  Note that the infix
directive in {\tt either-infixes.sml} is repeated, because {\tt local}
limits limits the scope of the directive.  Although the repetition is
unfortunate, it is preferable to repeat {\tt either-infixes.sml} in
{\tt either.mlb}, rather than require every client of {\tt either.mlb}
to also import {\tt either-infixes.sml}.

%
\subsec{Path maps}{mlb-path-map}
As noted above, paths can be either relative or absolute.  However,
using a fixed relative or absolute path to a library makes it
difficult to move either the client or the library.  Hence, MLton
allows path variables to appear in paths in the form {\tt \$(VAR)}.
The mapping from path variables to paths is initialized by reading two
configuration files: a system-wide one and a user-specific one.  The
system-wide configuration file is read from {\tt
/\prefix/lib/mlton/mlb-path-map}.  The user-specific configuration
file is read from {\tt .mlton/mlb-path-map} in the user's home
directory (which must be given by the {\tt HOME} environment
variable).  

The format of an {\tt mlb-path-map} file is a sequence of lines; each
line consists of two, white-space delimited tokens.  The first token
is a path variable {\tt VAR} and the second token is the path to which
the variable is mapped.  The path may include path variables, which
are recursively expanded.  Configuration files are processed from top
to bottom, system-wide before user-specific; later mappings take
precedence over earlier mappings.  The system-wide configuration file
makes the following path variables available:\\
\hspace*{0.5in}
\begin{tabular}{ll}
{\tt LIB\_MLTON\_DIR} & {\tt /\prefix/lib/mlton} \\
{\tt MLTON\_ROOT} & {\tt \$(LIB\_MLTON\_DIR)/sml}
\end{tabular}

%
\subsec{Available libraries}{mlb-libs}

\begin{tabular}{l}
{\tt \$(MLTON\_ROOT)/basis/basis.mlb} \\\hspace{0.5in}
The Standard ML Basis Library (see \secref{basis-library}). 
\\
{\tt \$(MLTON\_ROOT)/basis/pervasive-types.mlb} \\\hspace{0.5in}
The top-level types and constructors of the Basis Library (see \secref{basis-pervasive-types}).
\\
{\tt \$(MLTON\_ROOT)/basis/pervasive-exns.mlb} \\\hspace{0.5in}
The top-level exception constructors of the Basis Library (see \secref{basis-pervasive-exns}).
\\
{\tt \$(MLTON\_ROOT)/basis/pervasive-vals.mlb} \\\hspace{0.5in}
The top-level values of the Basis Library, without infix status (see \secref{basis-pervasive-vals}).
\\
{\tt \$(MLTON\_ROOT)/basis/overloads.mlb} \\\hspace{0.5in}
The top-level overloaded values of the Basis Library, without infix status (see \secref{basis-overloads}).
\\
{\tt \$(MLTON\_ROOT)/basis/equal.mlb} \\\hspace{0.5in}
The polymorphic equality {\tt =} and inequality {\tt <>} values, without infix status.
\\
{\tt \$(MLTON\_ROOT)/basis/infixes.mlb} \\\hspace{0.5in}
The infix declarations of the Basis Library.
\\
{\tt \$(MLTON\_ROOT)/basis/pervasive.mlb} \\\hspace{0.5in}
The entire top-level environment of the Basis Library, with infix status.
\\
{\tt \$(MLTON\_ROOT)/basis/basis-1997.mlb} \\\hspace{0.5in}
The (deprecated) 1997 specification of the Standard ML Basis Library.
\\
{\tt \$(MLTON\_ROOT)/basis/mlton.mlb} \\\hspace{0.5in}
The {\tt MLton} structure and signatures (see \secref{mlton-ext}).
\\
{\tt \$(MLTON\_ROOT)/basis/sml-nj.mlb} \\\hspace{0.5in}
The {\tt SMLofNJ} structure and signature (see \secref{sml-nj-ext}).
\\
{\tt \$(MLTON\_ROOT)/basis/unsafe.mlb} \\\hspace{0.5in}
The {\tt Unsafe} structure and signature (see \secref{unsafe-ext}).
\\
{\tt \$(MLTON\_ROOT)/mlyacc-lib/mlyacc-lib.mlb} \\\hspace{0.5in}
Modules used by parsers built with {\tt mlyacc}.
\\
{\tt \$(MLTON\_ROOT)/cml/cml.mlb} \\\hspace{0.5in}
Concurrent ML, a library for message-passing concurrency
(see {\tt \$(MLTON\_ROOT)/cml/README}).
\end{tabular}

%
\subsec{Annotations}{mlb-anns}

Annotations are a mechanism that allows a library author to control
options that affect the elaboration of SML source files.
Conceptually, a basis file is elaborated in a default annotation
environment (just as it is elaborated in an empty basis).  The {\tt
ann {\it anns} in {\it basdec} end} declaration merges the annotations
{\it anns} with the ``current'' annotation environment for the
elaboration of {\it basdec}.  

In the explanation below, for annotations that take a boolean argument
{\tt \trueFalse}, the first value listed is the default annotation.

\begin{description}
\option{allowExport \falseTrue}
If true, allow the {\tt \_export} expression form of \secref{export} to
appear in imported source files.

\option{allowImport \falseTrue}
If true, allow the {\tt \_import} expression form of \secref{import} to
appear in imported source files.

\option{forceUsed}
Force all identifiers in the basis denoted by the body of the {\tt
ann} to be considered used; use in conjunction with {\tt warnUnused
true}.

\option{sequenceUnit \falseTrue}
If true, then in the sequence expression {\tt (e1; e2)}, it is a type
error if {\tt e1} is not of type {\tt unit}.  This can be helpful in
detecting curried applications that are mistakenly not fully applied.
To silence spurious errors, you can use {\tt ignore e1}.

\option{warnMatch \trueFalse}
Report nonexhaustive and redundant matches.

\option{warnUnused \falseTrue}
Report unused identifiers.
\end{description}

%
\subsubsection{Examples}

We demonstrate how to use annotations in two common scenarios.

Programs that automatically generate source code can often produce
nonexhaustive matches, relying upon invariants of the generated code
to ensure that the matches never fail.  A programmer may wish to elide
the nonexhaustive match warnings from this code, in order that
legitimate warnings are not missed in a flurry of false positives.  To
do so, the programmer simply annotates the generated code with the
{\tt warnMatch false} annotation:
\begin{verbatim}
local
  $(GEN_ROOT)/gen-lib.mlb

  ann warnMatch false in
    foo.gen.sml
  end
in
  signature FOO
  structure Foo
end
\end{verbatim}

Standard ML libraries can delivered via {\tt .mlb} files.  Authors of
such libraries should strive to be mindful of the ways in which
programmers may choose to compile their programs.  For example,
although the defaults for {\tt sequenceUnit} and {\tt warnUnused} are
{\tt false}, periodically compiling with these annotations defaulted
to {\tt true} can help uncover likely bugs.  However, a programmer is
unlikely to be interested in unused modules from an imported library,
and the behavior of {\tt sequenceUnit true} may be incompatible with
some libraries.  Hence, a library author may choose to deliver a
library as follows:
\begin{verbatim}
ann 
  sequenceUnit false,
  warnMatch true,
  warnUnused true, forceUsed
in
  local
    file1.sml
    ...
    filen.sml
  in
    functor F1
    ...
    signature S1
    ...
    structure S1
    ...
  end
end
\end{verbatim}
The annotations {\tt sequenceUnit true} and {\tt warnMatch true} have
the obvious effect on elaboration.  The annotations {\tt warnUnused
true} and {\tt forceUsed} work in conjunction --- warning on any
identifiers that do not contribute to the exported modules, and
preventing warnings on exported modules that are not used in the
remainder of the program.  Many of the available libraries listed in
\secref{mlb-libs} are delivered with these annotations.
